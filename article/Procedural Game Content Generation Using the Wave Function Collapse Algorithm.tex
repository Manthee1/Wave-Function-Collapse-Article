\documentclass[10pt,twoside,a4paper]{article}
\usepackage[IL2]{fontenc} % lepšia sadzba písmena Ľ než v T1
\usepackage[utf8]{inputenc}
\usepackage{graphicx}
\usepackage{url} % príkaz \url na formátovanie URL
\usepackage{hyperref} % odkazy v texte budú aktívne (pri niektorých triedach dokumentov spôsobuje posun textu)

\usepackage{cite}

\pagestyle{headings}
\title{Procedural Game Content Generation Using the Wave Function Collapse Algorithm\thanks{Semestrálny projekt v predmete Metódy inžinierskej práce, ak. rok 2022/23, vedenie: Meno Priezvisko}}
\author{Martin Dinja\\[2pt]
	{\small Slovak University of Technology in Bratislava}\\
	{\small Faculty of Informatics and Information Technologies}\\
	{\small \texttt{xdinja@stuba.sk}}
}

\date{\small 30. september 2022}

\begin{document}

\maketitle

\tableofcontents

\begin{abstract}
    \begin{center}
        This thesis deals with the generation of game content using the Wave Function Collapse algorithm. 
    \end{center}
\end{abstract}

\section{Introduction}\label{sec:introduction}

Motivujte čitateľa a vysvetlite, o čom píšete. Úvod sa väčšinou nedelí na časti.

Uveďte explicitne štruktúru článku. Tu je nejaký príklad.
Základný problém, ktorý bol naznačený v úvode,a jeho teória je podrobnejšie vysvetlená v časti~\ref*{theory}
Dôležité súvislosti sú uvedené v častiach~\ref{dolezita} a~\ref{dolezitejsia}.
Záverečné poznámky prináša časť~\ref{zaver}.

\section{Theory}\label{sec:theory}
\cite{BL22}
\cite{CHF20}
\cite{KLL+19}
\cite{LRGC22}
\cite{NMBP20}
\section{Implementation}\label{sec:implementation}
\section{Results}\label{sec:results}
\section{Comparison with other methods}\label{sec:comparison}
\section{Conclusion}\label{sec:conclusion}

\bibliography{lit}
\bibliographystyle{alpha}

\end{document}
