\documentclass[10pt,twoside,a4paper]{article}
\usepackage[IL2]{fontenc} % lepšia sadzba písmena Ľ než v T1
\usepackage[utf8]{inputenc}
\usepackage{graphicx}
\usepackage{url} % príkaz \url na formátovanie URL
\usepackage{hyperref} % odkazy v texte budú aktívne (pri niektorých triedach dokumentov spôsobuje posun textu)

\usepackage{cite}

\pagestyle{headings}
\title{Procedural Game Content Generation Using the Wave Function Collapse Algorithm\thanks{Semestrálny projekt v predmete Metódy inžinierskej práce, ak. rok 2022/23, vedenie: Meno Priezvisko}}
\author{Martin Dinja\\[2pt]
	{\small Slovak University of Technology in Bratislava}\\
	{\small Faculty of Informatics and Information Technologies}\\
	{\small \texttt{xdinja@stuba.sk}}
}

\date{\small 30. september 2022}

\begin{document}

\maketitle

\tableofcontents

\begin{abstract}
    \begin{center}
        This thesis deals with the generation of game content using the Wave Function Collapse algorithm. 
    \end{center}
\end{abstract}

\section{Introduction}\label{sec:introduction}

% An Introduction to the application of the wave function collapse algorithm for procedural generation of game content.
Procedural content generation (PCG) is a general term for a system that follows some patterns and generates an output based on those patterns.
Its use case is generating assets or content that would be too time-consuming to create manually.
PCG is mainly connotatively tied to game content generation, but its use cases can be more creative.
Most PCG systems use game-specific assets with game-specific rules and algorithms to generate their content.
A more general application fit for a wide range of games is the Wave Function Collapse algorithm developed by Maxim Gumin \cite{WFC}.
Wave Function Collapse is a greedy PCG algorithm based on the concept of collapsing a wave function, which is a mathematical representation of a quantum state.
This method can generate a large, high-quality, consistent output from a small set of input patterns.
It can be used in various applications, although its most commonly used for generating 2D tilemaps for games.
It can also generate 3D models, music, poetry, and more.
WFC's output can only be as good as its input, so it's essential to have a good set of input patterns and rules. 
My goal in this thesis is to create a simple, configurable application that will use the WFC algorithm to generate tilemaps for games, given a user-defined set of input patterns and rules.


\section{Theory}\label{sec:theory}
As mentioned in the introduction [\ref*{sec:introduction}], the Wave Function Collapse algorithm is a greedy PCG algorithm based on the concept of collapsing a wave function, which is a mathematical representation of a quantum state.
A function starts in a superposition of values and collapses when that function is measured.
The result of the measurement is the value of the function at that point.
So applying that to the PCG of a 2D tilemap, first, we need to define a set of input patterns and their rules concerning each other.
Then the algorithm will assign each tile a superposition of all possible patterns.
After prepending the superpositions, the algorithm will assign one random pattern to a tile with the least superpositions, also known as the least entropy, effectively collapsing its wave function, which is also why it is called a collapse algorithm.
Then we reevaluate all the affected tiles and their possible patterns and remove the ones that don't fit the rules.
Then we repeat the process until there is no more entropy in the tilemap.

\section{Implementation}\label{sec:implementation}


\section{Results}\label{sec:results}
\section{Comparison with other methods}\label{sec:comparison}
\section{Conclusion}\label{sec:conclusion}
\cite{BL22}
\cite{CHF20}
\cite{KLL+19}
\cite{LRGC22}
\cite{NMBP20}
\section{Implementation}\label{sec:implementation}
\section{Results}\label{sec:results}
\section{Comparison with other methods}\label{sec:comparison}
\section{Conclusion}\label{sec:conclusion}

\bibliography{lit}
\bibliographystyle{alpha}

\end{document}
